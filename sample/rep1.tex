\documentclass[12pt]{jarticle} %記事や論文を書く際にjarticleを使用

%枠付文章とpng画像を出力するため
\usepackage{ascmac}%枠付き文字出力
\usepackage{fancybox}
\usepackage{framed}
\usepackage[dvipdfmx]{graphicx}%pdf出力のための
\usepackage{mediabb}
\usepackage{color}


%\usepackage{listings,jlisting}.cをpdfに出力
\usepackage{listings, jlisting}
\lstset{
  language={java},%言語名前
  basicstyle={\footnotesize\ttfamily},%ソースコードのフォント設定
  identifierstyle={\small},
  ndkeywordstyle={\small},
  keywordstyle={\ttfamily\color[cmyk]{0,1,0,0}},
  stringstyle={\small\ttfamily\color[rgb]{0,0,1}},
  frame={tb},
  breaklines=true,%行が長くなった際の改行の有無
  xrightmargin=0zw,%右の余白の大きさ
  xleftmargin=3zw,%左の余白の大きさ
  numbers=left,%行番号表示場所
  stepnumber=1,%行番号の増分
  numbersep=1zw,%行番号と本文の間隔
  numberstyle=\ttfamily,
  frame=tRBl,
  framesep=5pt,
  commentstyle={\ttfamily\color[cmyk]{1,0.4,1,0}},%コメントアウトのフォント設定
  flexiblecolumns = true,
  classoffset=1,
  showstringspaces=false,
  tabsize=4
}


%本文領域を広め(空白箇所マージン領域を小さめ)に設定
\setlength{\oddsidemargin}{0mm}%文字の左からの間隔
\setlength{\topmargin}{-13mm}%用紙の上からの間隔
\setlength{\headheight}{12pt}%ヘッダーの高さ
\setlength{\headsep}{25pt}%ヘッダーとボディとの間隔
\setlength{\textwidth}{160mm}%文字出力の横幅を設定
\setlength{\textheight}{230mm}%文字出力の縦幅を設定
\setlength{\marginparwidth}{20pt}%注釈用の余白
\setlength{\footskip}{25pt}%フッターとボディとの間隔
\setlength{\evensidemargin}{-1cm}
\thispagestyle{empty}

\begin{document}

\begin{titlepage}
    \begin{center} %vspaceは縦方向のスペースで,hspaceは横方向のスペース
        \fontsize{25pt}{0pt}\selectfont %フォントサイズと行送りを設定
        \vspace*{100truept}
        \bf{ProgrammingII}\\ %タイトル1行目
        \vspace*{10truept}
        \bf{Report  \#1}\\ %タイトル2行目
        \vspace{340truept}
    \end{center}
    \begin{flushright}
        {\large
            \fontsize{16pt}{0pt}\selectfont
          所  属  :琉球大学 工学部 情報工学科\\
            \vspace{5truept}
          学籍番号:155730B\\
            \vspace{5truept}
          氏  名  :清水 隆博\\
            \vspace{5truept}
          提 出 日:2015年12月21日\\
            \vspace{5truept}
        }    
    \end{flushright}
\end{titlepage}

\tableofcontents %目次を自動作成するコマンド
\setcounter{page}{1} %ページ数を1とする
\pagestyle{plain}
%事前にjlisting.styをlistingにぶちこみ,sudo mktexlsr で更新を行う。

\newpage

\begin{flushleft}

\section{ソースコードの解説}
\subsection{Corners.java}
\lstinputlisting{Corners.java}

\newpage

\subsection{Crazy.java}
\subsubsection{ソースコード}
\lstinputlisting{Crazy.java}

\newpage

\subsection{Fire.java}
\subsubsection{ソースコード}
\lstinputlisting{Fire.java}

\newpage

\subsection{Interactive.java}
\subsubsection{ソースコード}
\lstinputlisting{Interactive.java}
\subsubsection{考察}
\subsubsection{メリット}
\begin{enumerate}
\item hoge
\end{enumerate}
\subsubsection{デメリット}
\begin{enumerate}
\item hoge
\end{enumerate}

\subsection{Interactive\_v2.java}
\subsubsection{ソースコード}
\lstinputlisting{Interactive_v2.java}
\subsubsection{考察}
\subsubsection{メリット}
\begin{enumerate}
\item hoge
\end{enumerate}
\subsubsection{デメリット}
\begin{enumerate}
\item hoge
\end{enumerate}

\subsection{MyFirstJuniorRobot.java}
\subsubsection{ソースコード}
\lstinputlisting{MyFirstJuniorRobot.java}
\subsubsection{考察}
\subsubsection{メリット}
\begin{enumerate}
\item hoge
\end{enumerate}
\subsubsection{デメリット}
\begin{enumerate}
\item hoge
\end{enumerate}

\subsection{PaintingRobot.java}
\subsubsection{ソースコード}
\lstinputlisting{PaintingRobot.java}
\subsubsection{考察}
\subsubsection{メリット}
\begin{enumerate}
\item hoge
\end{enumerate}
\subsubsection{デメリット}
\begin{enumerate}
\item hoge
\end{enumerate}

\subsection{RamFire.java}
\subsubsection{ソースコード}
\lstinputlisting{RamFire.java}
\subsubsection{考察}
\subsubsection{メリット}
\begin{enumerate}
\item hoge
\end{enumerate}
\subsubsection{デメリット}
\begin{enumerate}
\item hoge
\end{enumerate}

\subsection{SittingDuck.java}
\subsubsection{ソースコード}
\lstinputlisting{SittingDuck.java}
\subsubsection{考察}
\subsubsection{メリット}
\begin{enumerate}
\item hoge
\end{enumerate}
\subsubsection{デメリット}
\begin{enumerate}
\item hoge
\end{enumerate}

\subsection{SpinBot.java}
\subsubsection{ソースコード}
\lstinputlisting{SpinBot.java}
\subsubsection{考察}
\subsubsection{メリット}
\begin{enumerate}
\item hoge
\end{enumerate}
\subsubsection{デメリット}
\begin{enumerate}
\item hoge
\end{enumerate}

\subsection{Target.java}
\subsubsection{ソースコード}
\lstinputlisting{Target.java}
\subsubsection{考察}
\subsubsection{メリット}
\begin{enumerate}
\item hoge
\end{enumerate}
\subsubsection{デメリット}
\begin{enumerate}
\item hoge
\end{enumerate}

\subsection{TrackFire.java}
\subsubsection{ソースコード}
\lstinputlisting{TrackFire.java}
\subsubsection{考察}
\subsubsection{メリット}
\begin{enumerate}
\item hoge
\end{enumerate}
\subsubsection{デメリット}
\begin{enumerate}
\item hoge
\end{enumerate}

\subsection{Tracker.java}
\subsubsection{ソースコード}
\lstinputlisting{Tracker.java}
\subsubsection{考察}
\subsubsection{メリット}
\begin{enumerate}
\item hoge
\end{enumerate}
\subsubsection{デメリット}
\begin{enumerate}
\item hoge
\end{enumerate}

\subsection{VelociRobot.java}
\subsubsection{ソースコード}
\lstinputlisting{VelociRobot.java}
\subsubsection{考察}
\subsubsection{メリット}
\begin{enumerate}
\item hoge
\end{enumerate}
\subsubsection{デメリット}
\begin{enumerate}
\item hoge
\end{enumerate}

\subsection{Walls.java}
\subsubsection{ソースコード}
\lstinputlisting{Walls.java}
\subsubsection{考察}
\subsubsection{メリット}
\begin{enumerate}
\item hoge
\end{enumerate}
\subsubsection{デメリット}
\begin{enumerate}
\item hoge
\end{enumerate}

\section{各ロボットの対戦}

\subsection{総当り戦}
\begin{itemize}
\item 以下に総当りの票を示す。
\item 今回の対戦ルールは下記の様に行った
\end{itemize}

\hrulefill

\begin{enumerate}
\item 使用バトルフィールドの大きさ
\end{enumerate}

\end{flushleft}
%番号付き箇条書き
%\begin{enumerate}
%\item 1つめ


\begin{thebibliography}{9} 
	\bibitem{}Java言語プログラミングレッスン
	\bibitem{}javaの教科書
\end{thebibliography}
\end{document}